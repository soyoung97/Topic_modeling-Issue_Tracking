\section{Off-issue tracking}
For off-issue tracking, we first categorize topics given as Trend analysis part.
In this section, we denote a document as sequence of tokens plus its created time $\mathbb{D} := (\Sigma^{+}, t)$, when $t \in \mathbb{R}$ (timestamp of creation time).
and the set of document of topic $a$ as $\mathbb{T}_{a} \in \mathcal{P}(\mathbb{D})$.

\subsection{BoW extractuion}
In first, we have to extract document in some space which we can analyze quantatively.
We use BoW as morphism from document space to vector space $\mathbb{R}^{N}$, which we can
analyze similarity of document. In addition, we add one more dimension to give information
of document creation time. From pre-calculated set of tokens $\Sigma := \{ \sigma_{1}, \sigma_{2}, \ldots, \sigma_{n} \}$,
our transformation $b: \mathbb{D} \rightarrow \mathbb{R}^{n+1}$ is defined inductively as
\[
\begin{cases}
    b([], t) := t * e_{n+1}\\
    b(\sigma_{i} :: tl, t) := e_{i} + b(tl, t)
\end{cases} 
\]
Then, morphism from $\mathbb{T}_{a} \in \mathcal{P}(\mathbb{D})$ to $\mathcal{P}(\mathbb{R}^{n+1})$ is naturally induced from $b$ as
$\phi(\mathbb{T}_{a}) = \{b(d) | d \in \mathbb{T}_{a}\}$

\subsection{Relation between semantic of document and BoW}
We know that there are documents and events which have similar meaning, but we cannot formalize it because we currently do not have
model of language interpretation in metric space. But we can assume \textit{such} space exists, i.e. there is an isomorphism
$\phi: \mathbb{D} \rightarrow \mathbb{D}^{\#}$, when $(D^{\#}, d^{\#})$ is metric space. It is not hard to assume this structure,
since similar concept is already introduced as Entity comparison/Behavior comparison operator of Semantic algebra \cite{wang2013semantic}.

Our desired result is that $b$ with euclidean distance successfuly models $(D^{\#}, d^{\#})$, but we cannot show it because we do not have
constructive definition of $D^{\#}$. But if it has sufficient approximation, (bounded approximation)
We can derive more interesting properties (such as bounded error from BoW to Event space, etc).

\begin{definition}
    $b$ has approximation of $\phi$ with bound $K, \epsilon$ iff there exists an Lipshitz continuous $\pi$ with $K$ that $d^{\#}(\pi(b(d)), \phi(d)) \leq \epsilon$.
\end{definition}

\subsection{Relation between semantic of event and BoW}
Once semantic of document is defined, we can build similar notion of event as metric space. To build such space, we first understand about
relation between document and event.
\begin{itemize}
    \item similar document refer similar event.
    \item similar event (even same event) may be refered by documents with far distance, but it is not arbitrarly far.
\end{itemize}
we can formulize this as logical formlua, with definition of $e: D^{\#} \rightarrow E^{\#}$. ($(E^{\#}, e^{\#})$ is metric space for event)
\begin{itemize}
    \item if $d^{\#}(d_{1}, d_{2})$ is sufficiently small, then $e^{\#}(e(d_{1}), e(d_{2}))$ is sufficiently small.
    \item when $e^{\#}(e(d_{1}), e(d_{2}))$ is small, it doesn't mean $d^{\#}(d_{1}, d_{2})$ is small but is bounded.
\end{itemize}
begin with this fact, we can find very interesting property which generalize this: continuity.
\begin{definition}
    $e$ is Lipschitz continuous with $K$ if and only if\\ $e^{\#}(e(d_{1}), e(d_{2})) \leq K d^{\#}(d_{1}, d_{2})$.
\end{definition}

We can check that if $e$ is Lipschitz continuous with $K_{e}$, then above two property is satisfied. Also, it derives important fact:
If we have approximation of semantics with bounded error, then there also exists approximation of event with bounded error.
\begin{theorem}
    $b$ has approximation of $\phi$ with bound $K, \epsilon$, then there exists $\pi_{e}: \mathbb{R}^{n+1} \rightarrow E^{\#}$ s.t.
    $e^{\#}(\pi_{e}(b(d)), e(\phi(d))) \leq K_{e} \cdot \epsilon$. (it means $b$ has approximation of $e \cdot \phi$ with bound $K, K_{e} \cdot \epsilon$)
\end{theorem}
Although proof is directly derived from Lipschitz continuity, it emphasizes that if we have bounded approximate of document, then it guarantees
bounded approximation of event.

\subsection{Event clustering}

In this assumption about semantic of document ans event, we can build event clustering method. Before using techniques in $R^{n+1}$, we
focus on how this clustering in $R^{n+1}$ effects in $E^{\#}$.

\begin{theorem}
    if $b$ has approximation of $e \cdot \phi$ with bound $K, \epsilon$, then $e^{\#}(e \cdot \phi(d_{1}), e \cdot \phi(d_{2})) \leq 2 \cdot \epsilon + K || b(d_{1}) - b(d_{2}) || $.
\end{theorem}

\begin{proof}
    \begin{align*}
     & e^{\#}(e \cdot \phi(d_{1}), e \cdot \phi(d_{2})) \leq e^{\#}(e \cdot \phi(d_{1}), \pi_{e}(b(d_{1}))) + \\
     & e^{\#}(\pi_{e}(b(d_{1})), \pi_{e}(b(d_{2}))) + e^{\#}(\pi_{e}(b(d_{2})), e \cdot \phi(d_{2})) \leq \\
     & \epsilon + e^{\#}(\pi_{e}(b(d_{1})), \pi_{e}(b(d_{2}))) + \epsilon \leq \\
     & 2 \cdot \epsilon + K || b(d_{1}) - b(d_{2}) ||.
    \end{align*}
\end{proof}

It shows that, if we make good Vector transformation $b$, then it automatically guarantees bounded error for distance of extracted event, without
construction of $\pi$, $\phi$, $e$ or any other. Begin with this fact, we derive constructive definition of partition for documents using approximated
transformation $b$. To do that, we first define similarity for two documents.

\begin{definition}[Similarity relation]
    $\approx_{\mathbb{R}^{n+1}, \delta} \in \mathcal{P}(\mathbb{D \times D})$ is defined as \begin{displaymath}
        d_{1} \approx_{\mathbb{R}^{n+1}, \delta} d_{2} \Longleftrightarrow || b(d_{1}) - b(d_{2}) || \leq \delta.
    \end{displaymath}
    Similarly, $\approx_{E^{\#}, \delta} \in \mathcal{P}(\mathbb{D \times D})$ is defined as \begin{displaymath}
    d_{1} \approx_{E^{\#}, \delta} d_{2} \Longleftrightarrow e^{\#}(e \cdot \phi(d_{1}), e \cdot \phi(d_{2})) \leq \delta.
    \end{displaymath}
\end{definition}
then $\approx_{\mathbb{R}^{n+1}, \delta} \subseteq \approx_{E^{\#}, 2 \cdot \epsilon + K \cdot \delta}$ holds by above theorem. Thus it is quite
reasonable to use $\approx_{\mathbb{R}^{n+1}, \delta}$ to cluster events.

\begin{definition}[Transitive closure]
    $\approx_{\mathbb{R}^{n+1}, \delta}^{*}$ is smallest relation on $\mathbb{D}$ that contains $\approx_{\mathbb{R}^{n+1}, \delta}$ and is transitive.
\end{definition}

Then $\approx_{\mathbb{R}^{n+1}, \delta}^{*}$ is reflexive, symmetric and transitive, which can be equivalence relation. Then, we can partition
documents with this equivalence relation.

\begin{definition}[Partiton of $\mathbb{D}$]
    when $\approx$ is equivalence relation, $\mathbb{D}/\approx := \{[a] | a \in \mathbb{D}\}$, when $[a] := \{b \in \mathbb{D} | a \approx b \}$.
\end{definition}

By substitute $\mathbb{D}$ to $\mathbb{T}_{a}$, finally we have $\mathbb{T}_{a}/\approx_{\mathbb{R}^{n+1}, \delta}^{*}$ as successful approximation 
of event partition of topic $a$. Now, we are going to explain how most relevent description of event is extracted from each partiton.