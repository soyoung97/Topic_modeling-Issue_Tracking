\section{Introduction}

Online contents has grown big recently,
and people are viewing more news on the internet.
However, due to the heavy amount of data(news),
it has a limitation to categorize the news manually.
Also, it is almost impossible to track the events
related to the issue while looking all the articles.

Looking at all yearly issues,
we can see that the news articles can be clusted
into some issues.
For example, in the case of the year 2017,
there was a lot of news related with the former president
and her political crimes.
The term \textit{Trend Analysis} means
clustering those kind of news articles and analyze the clusters. 
And, for each issue, we can see the \textit{events} related to the issue
and we can make the timeline of the main events for an issue.
In detail, there are two kinds of event: the first one is
the event which is directly related to an issue,
and the second one is not directly linked, but topically related issue.
In the paper, we call the first one as \textit{``on-issue event''},
and the second one as \textit{``off-issue event(related-issue event)''}

In this paper, we suggest a method to analyze the trends
and track the events by three steps: 1. \textit{Trend Analysis},
2. \textit{On-issue Event Tracking}, and 3. \textit{Off-issue Event Tracking}.

After doing all the progress,
we could analyze the yearly trends of Korea from 2015 to 2017. 
Also, we tried to track down some important issues extracted from above.
We captured four quarterly events for each issue,
and also tracked several off-issue events.

Our research is important and effective because we minimized the human(manual) efforts
throughout the progress, for summarizing the 270K news articles.
The methods we suggest in this paper can be widely used in
yearly trend analysis, not just news, but marketing, research, or the other fields as well.