\section{Trend analysis}
\usepackage{minted}

\subsection{Data Preprocessing}
\subsubsection{Data Format}
As described in the READ.ME of data provided, The targeted data is from the Korean Herald, National Section news. The period of the dataset is from 2015 to 2017. The Crawled date of the dataset is 2018-10-26. Data format is Json, and there are total of 6 data headers - title, author, time, description, body, and section. Total of 23769 news are included in this dataset.
\subsubsection{Load Data}
In order to load the data, the instructions recommended at READ.ME are followed. Pandas library is used for better storing and access of the news text.
\subsubsection{Libraries Used}
For this project, we used pandas and gensim python libraries.


\subsection{Previous approaches}
Issue trend analysis can be seen as a part of Topic modeling. By searching fields of recent Topic modeling, LDA has shown to have good performance. As a result, LDA is used as a baseline algoritm for this project.
A recent study(2018) on Topic Modeling shows that Topic Quality improves when Named Entities are promoted.\site{krasnashchok-jouili-2018-improving} This paper proposes 2 techniques: 1.Independent Named Entity Promoting and 2.Document Dependent Named Entity Promoting. Independent Named Entity Promoting promotes the importance of the named entities by applying scalar multiplicaion alpha to the importance of the named entity word. Document Dependent Named Entity Promoting promotes the importance of the named entities by setting the weights of the named entities as maximum term-frequency per document. For Independent Named Entity Promoting, the value of alpha can be changed flexibily, but results conducted by this paper shows that setting alpha as 10 showed the best results.
We take advantage of this paper and implement Named Entity Promoted Topic Modeling done by LDA.
\subsection{Experiments}
\subsubsection{Data Tokenization}
\subsubsubsection{Lemmatization is not always good}
At first try, Lemmatization(converting words into base forms) and removal of stopwords were conducted before we run the LDA algorithm and extract Named Entities. We thought that converting words into base forms and reducing the total vocabulary size would increase the performance of topic modeling. Stopwords were taken from \mintinline{python}{nltk.corpus.stopwords.words("english")}, and lemmatization functon was taken from \mintinline{python}{gensim.utils.lemmatize}. \mintinline{python}{res.append(lemmatize(raw\_text, stopwords=stopwords))}. But after we do lemmatization, remove stopwords, and tokenize the data, no Named Entities were extracted from the preprocessed corpus. We think the reason for this is as follows. 
First, words are all converted into lower case when we do lemmatization. This makes the Named Entitiy Recognition system(NER system) to work poorly because we have removed the original information whether the word has a high probability that it is a "Proper pronoun" or not.(고유대명사).
Second, words are transformed into their base forms, limiting NER system to detect specific words. There also could be cases that the words are transformed into meanings other then their original meanings. For example, "Cooking" and "Cooker" are both converted into "cook" when they are lemmatized, and this makes the word to lose the original information.
Third, original relationships between words are lost, because of the removal of stopwords. When we do NER, we have to do the POS tagging of the sentence and then input both the word sequence and the POS sequence of the text. But when we artificially remove stopwords and then do NER, original relationships between words are disrupted and broken. This limits NER system to perform well.

For these 3 reasons, we decided to NOT apply lemmatization for tokenization, because lemmatization lose so much information about the original text and disrupts the NER system's ability to detect
 Named Entities properly. We decided to just do POS tagging and then do NER. We just used word\_tokenize from nltk.tokenize.

\subsubsection{Extract NER}
By using ne\_chunk from nltk and pos\_tag from nltk.tag, we extracted Named entities from the original news dataset. NER also extracts multi-word information of Named Entities other than just class
ifying whether a word is a named entity or not, so we decided to use that information. We store single-word Named Entities and multi-word named entities separately. As a result, NER and multi-wor
d extraction of NER are both processed.

below figure is the topic modeling result(of all time lengths from 2015 to 2017) WITH NER Promoting and WITHOUT NER Promoting. We can see the difference between those two results, and we can conc
lude topic modeling with NER promoting shows better performance.
\begin{figure*}[t]
    \centering
\includegraphics[width=0.8\textwidth]{afterner.png}
\includegraphics[width=0.8\textwidth]{beforener.png}
\end{figure*}
\subsubsection{Do LDA}
At first try, we ran LDA on naive ner boosted news dataset. but with this approach, we found out that stopwords are classified as top(important)words according to the result of LDA. So we decided to remove stopwords after all the preprocssing(including NER weight promoting)are done. The timing of removal of stopwords are important, as removing stopwords before NER will affect the NER result. Stopword removing are done right before feeding the tokens into LDA. After the removal of stopwords, we could see that the results were much better.

\subsubsection{Apply neuroNER} 
On the topic modeling paper that we referenced says that it uses neuroNER. neuronNER is an easy-to-use program for named entity recognition based on neural networks presented in emnlp 2017. \site{2017neuroner} This neuroNER tool is trained on CONLL2003 dataset and recognizes four types
of NE: person, location, organization and miscellaneous. Instead of using \mintinline{python}{ne_chunk(pos_tag(preprocessed_text), binary=True)}, we use
\begin{minted}{python}
nn = neuromodel.NeuroNER(train_model=False, use_pretrained_model=True)
nn.predict(preprocessed_text)
\end{minted}
to extract Named Entities from the text.


\subsubsection{Do LDA with NER promoting.}
First, split the dataset each year. Then, get tokens for each document with promoted NER frequency (X 10). With this corpus, run the LdaModel with num\_topics of 10 and num\_words of 30 to 50.
\subsubsubsection{Tuning LDA hyperparameters}
At first, we decided to train the LDA model with num\_topics of 10 and num\_words of 15. But the results were not very explainable. After experimenting with num\_topics and num\_words, we found that setting num\_topics of 10 is the best representative of the total news. Also, since the only removed word was the stop word, non-ascii character, or unrelated words such as \mintinline{python}{" \xec ' ( ) . ,} were introduced in the topic result. To extract useful information, we increased num\_words for each topics to 50.


